% Preambule.tex
% Packages de base
\usepackage[utf8]{inputenc}
\usepackage[T1]{fontenc}
\usepackage{pdfpages}
\usepackage[english]{babel} % Langue française
\usepackage{amsmath, amssymb} % Mathématiques
\usepackage{graphicx} % Insertion d'images
\usepackage{hyperref} % Hyperliens
\usepackage{geometry} % Mise en page
\usepackage{fancyhdr} % Pour personnaliser l'en-tête et le pied de page
\usepackage{float}
\usepackage{titlesec}
\usepackage{caption}


% Configurations spécifiques
\geometry{a4paper, margin=2.5cm} % Marges de la page
\setlength{\parskip}{\baselineskip} % Espacement entre les paragraphes
\setlength{\parindent}{0pt} % Pas d'indentation au début des paragraphes

\titleformat{\chapter}[hang]  % Utilisation du format 'hang' pour que tout soit sur la même ligne
  {\normalfont\huge\bfseries} % Style du titre (police normale, taille énorme, gras)
  {Chapter \thechapter\ \, :\ } % Format "Chapter" suivi du numéro du chapitre et des deux-points avec un espace après
  {0pt}                        % Pas d'espace supplémentaire
  {\huge}                       % Taille du texte du titre du chapitre