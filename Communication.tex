\chapter{Communication unit}
\thispagestyle{fancy}


\section{Energy For Connected Oject}

\subsection{Descriptive}

As part of the Innovative Smart System specialization in the 5th year, the course titled Energy for Connected Objects focuses on energy solutions for connected objects (IoT), addressing innovative techniques such as ambient energy harvesting and wireless energy transfer. The objective of this course is to provide students with an in-depth understanding of modern methods for managing and generating energy for connected objects. These systems are crucial in a context where energy autonomy and sustainability are essential for the development of the Internet of Things.

The course is structured with a combination of lectures and practical work. The four theoretical sessions cover topics ranging from electricity generation and storage to ambient energy harvesting techniques (such as light, vibrations, and electromagnetic fields) and advanced wireless energy transfer methods (including near-field and far-field electromagnetic solutions). These concepts help to understand how to power connected objects wirelessly and without batteries, by exploiting available energy sources in the environment.

The two practical sessions allow students to apply these concepts by designing an emulator for connected objects and characterizing an electromagnetic energy harvesting system. Evaluation is based on a group lab report and an in-depth study of energy autonomy options for an innovative project. This study includes exploring battery-free power supply possibilities, paving the way for sustainable IoT projects. Additionally, the course provides a series of academic resources to further knowledge in energy harvesting and wireless transfer, which are essential for developing efficient and environmentally responsible IoT devices.


\subsection{Technical}

The technical dimension of this course focuses on two key areas: ambient energy harvesting and wireless energy transfer, with particular attention to practical aspects and technical challenges in designing energy-autonomous IoT systems.

Ambient Energy Harvesting
This technique involves capturing energy from the environment and converting it into electricity to power connected objects. Ambient energy sources include:

\begin{itemize}
    \item Light: primarily through photovoltaic cells that capture both solar and artificial light.
    \item Mechanical (kinetic): using electromagnetic, piezoelectric, or electrostatic transducers capable of converting vibrations or motion into electrical energy.
    \item Thermal: using thermal transducers like thermoelectric devices, which generate electricity from temperature gradients.
    \item Electromagnetic: using capacitive and inductive transducers to capture ambient electromagnetic fields.
\end{itemize}


This approach allows for continuous power, although it depends on the proximity of field sources. Each type of energy harvesting presents challenges related to availability, stability, and energy efficiency. Optimization techniques such as using hybrid transducers or transducer networks (series/parallel topologies) can improve the efficiency of energy collection for low-power IoT devices.

Wireless Energy Transfer
This field covers methods that enable the transmission of energy without cables, focusing on powering connected objects via dedicated energy sources, which are often more reliable and predictable than ambient sources. The main wireless energy transfer techniques include:

\begin{itemize}
\item Optical (light): using lasers or infrared for long-distance transfers, requiring a direct line of sight.
\item Acoustic: using ultrasound, especially for specific applications and short-range transfer.
\item Electromagnetic: which is divided into near-field transfer (capacitive or inductive coupling, suitable for short distances with high efficiency) and far-field transfer (radio or microwave waves), enabling longer ranges but typically with lower efficiency.
\end{itemize}


To maximize the energy efficiency of these wireless systems, devices like rectennas (rectifying antennas) are used, converting RF waves into DC current. Optimal rectenna design involves selecting an appropriate frequency, using impedance matching circuits, and carefully configuring the rectifier to minimize losses. Signal modulation techniques, impedance management, and waveform optimization are essential to reduce energy dissipation and meet the power and autonomy requirements of connected objects.

This course provides a technical understanding of energy management solutions for autonomous IoT systems, covering the design of autonomous sensors, reducing energy consumption, and best practices to optimize the efficiency of energy harvesting and transfer devices. These skills are crucial in the development of sustainable IoT applications, ranging from smart homes to industrial sensor networks.


\subsection{Analytical}

\section{Wireless Sensor Network}

\subsection{Descriptive}
\subsection{Technical}
\subsection{Analytical}

\section{5G technologies}

\subsection{Descriptive}

The course titled "From 3G to 6G", taught by Professor Étienne Sicard at INSA Toulouse, delves into the evolution of mobile technologies, from their inception to the cutting-edge advancements shaping the future. Aimed at fifth-year students, this course provides a comprehensive understanding of the technical, economic, and environmental aspects of mobile networks. It highlights the major technological transitions that have defined each generation and explores innovations paving the way toward the 6G era.

The course covers several key themes. The first section focuses on the miniaturization of electronic components, discussing advances in transistor design, including the shift from traditional MOSFETs to modern architectures like FinFETs and Nano-Sheet FETs. This evolution is framed within the broader context of the global transformation of the electronics industry and its impact on mobile network performance.

The second section revisits the history of mobile generations, from 3G to 6G, emphasizing key technological milestones and applications associated with each era. The 3G era, marked by the introduction of UMTS, was pivotal in enabling mobile data services. Subsequently, 4G introduced LTE, revolutionizing multimedia applications by providing significantly higher bandwidth. The 5G era brought the integration of the Internet of Things (IoT) and the utilization of millimeter-wave frequency bands, dramatically enhancing network capacity and speed.

Lastly, the course reflects on emerging technologies and the future of telecommunications, particularly through the roadmaps toward 6G outlined by industry leaders like Nokia, Huawei, and Orange. These roadmaps introduce transformative innovations such as Zero-Energy Devices and cyber-physical systems, which are poised to redefine connectivity.
\subsection{Technical}

The technical aspect of the course immerses students in the scientific foundations and innovations underpinning mobile telecommunications. A major focus is on technological miniaturization, which has revolutionized electronic device design over the years. With the continuous reduction in transistor size, evolving from traditional MOSFETs to FinFETs and Nano-Sheet FETs, the performance of electronic chips has greatly improved. These advances have enabled the production of more compact and energy-efficient devices capable of handling increasingly large volumes of data. For instance, processors like the Qualcomm Snapdragon 888 exemplify these advancements by incorporating cutting-edge artificial intelligence and 5G connectivity features.

The course also explores the evolution of mobile generations, shedding light on the bandwidth requirements and modulation technologies specific to each generation. The 3G era marked the advent of mobile internet with UMTS, while 4G, through LTE, revolutionized the use of multimedia applications with higher data speeds. The 5G era introduced millimeter-wave frequencies (26 GHz) and advanced technologies such as massive MIMO and beamforming, paving the way for sophisticated applications like autonomous vehicles and connected healthcare. Looking ahead, 6G, anticipated by 2030, aims to exploit even higher frequencies, exceeding 100 GHz, enabling ultra-fast data rates and minimal latency. However, these advancements come with significant technical challenges, including signal attenuation at high frequencies and increased energy consumption.

The exploration of ultra-high frequencies (UHF and beyond) is a central theme of the course. While 5G primarily operates between 3.4 GHz and 26 GHz, 6G aims to expand into frequencies exceeding 100 GHz, encompassing the W and D bands (130–174 GHz). This transition toward higher frequencies unlocks unique opportunities for applications such as augmented reality, telemedicine, and global connectivity via satellite constellations. However, it also introduces challenges related to wave propagation and the need for more efficient antenna designs.

Finally, the course provides an in-depth analysis of technologies and visions for 6G as envisaged by companies like Nokia and Huawei. These players emphasize innovative concepts such as Zero-Energy Devices, capable of operating solely on ambient energy, and integrated cyber-physical systems. These perspectives underscore the industry's commitment to balancing technological performance with environmental sustainability, a critical challenge for the future of telecommunications.

\subsection{Analytical}


