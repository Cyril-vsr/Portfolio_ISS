\chapter{Communication}

\section{Energy For Connected Oject}

\subsection{Descriptive}

As part of the Innovative Smart System specialization in the 5th year, the course titled Energy for Connected Objects focuses on energy solutions for connected objects (IoT), addressing innovative techniques such as ambient energy harvesting and wireless energy transfer. The objective of this course is to provide students with an in-depth understanding of modern methods for managing and generating energy for connected objects. These systems are crucial in a context where energy autonomy and sustainability are essential for the development of the Internet of Things.

The course is structured with a combination of lectures and practical work. The four theoretical sessions cover topics ranging from electricity generation and storage to ambient energy harvesting techniques (such as light, vibrations, and electromagnetic fields) and advanced wireless energy transfer methods (including near-field and far-field electromagnetic solutions). These concepts help to understand how to power connected objects wirelessly and without batteries, by exploiting available energy sources in the environment.

The two practical sessions allow students to apply these concepts by designing an emulator for connected objects and characterizing an electromagnetic energy harvesting system. Evaluation is based on a group lab report and an in-depth study of energy autonomy options for an innovative project. This study includes exploring battery-free power supply possibilities, paving the way for sustainable IoT projects. Additionally, the course provides a series of academic resources to further knowledge in energy harvesting and wireless transfer, which are essential for developing efficient and environmentally responsible IoT devices.


\subsection{Technical}

The technical dimension of this course focuses on two key areas: ambient energy harvesting and wireless energy transfer, with particular attention to practical aspects and technical challenges in designing energy-autonomous IoT systems.

Ambient Energy Harvesting
This technique involves capturing energy from the environment and converting it into electricity to power connected objects. Ambient energy sources include:

\begin{itemize}
    \item Light: primarily through photovoltaic cells that capture both solar and artificial light.
    \item Mechanical (kinetic): using electromagnetic, piezoelectric, or electrostatic transducers capable of converting vibrations or motion into electrical energy.
    \item Thermal: using thermal transducers like thermoelectric devices, which generate electricity from temperature gradients.
    \item Electromagnetic: using capacitive and inductive transducers to capture ambient electromagnetic fields.
\end{itemize}


This approach allows for continuous power, although it depends on the proximity of field sources. Each type of energy harvesting presents challenges related to availability, stability, and energy efficiency. Optimization techniques such as using hybrid transducers or transducer networks (series/parallel topologies) can improve the efficiency of energy collection for low-power IoT devices.

Wireless Energy Transfer
This field covers methods that enable the transmission of energy without cables, focusing on powering connected objects via dedicated energy sources, which are often more reliable and predictable than ambient sources. The main wireless energy transfer techniques include:

\begin{itemize}
\item Optical (light): using lasers or infrared for long-distance transfers, requiring a direct line of sight.
\item Acoustic: using ultrasound, especially for specific applications and short-range transfer.
\item Electromagnetic: which is divided into near-field transfer (capacitive or inductive coupling, suitable for short distances with high efficiency) and far-field transfer (radio or microwave waves), enabling longer ranges but typically with lower efficiency.
\end{itemize}


To maximize the energy efficiency of these wireless systems, devices like rectennas (rectifying antennas) are used, converting RF waves into DC current. Optimal rectenna design involves selecting an appropriate frequency, using impedance matching circuits, and carefully configuring the rectifier to minimize losses. Signal modulation techniques, impedance management, and waveform optimization are essential to reduce energy dissipation and meet the power and autonomy requirements of connected objects.

This course provides a technical understanding of energy management solutions for autonomous IoT systems, covering the design of autonomous sensors, reducing energy consumption, and best practices to optimize the efficiency of energy harvesting and transfer devices. These skills are crucial in the development of sustainable IoT applications, ranging from smart homes to industrial sensor networks.


\section{Analytical}



